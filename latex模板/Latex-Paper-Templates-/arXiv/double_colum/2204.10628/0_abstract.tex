Knowledge-intensive language tasks require NLP systems to both provide the correct answer and retrieve supporting evidence for it in a given corpus. 
Autoregressive language models are emerging as the de-facto standard for generating answers, with newer and more powerful systems emerging at an astonishing pace.
In this paper we argue that all this (and future) progress can be directly applied to the retrieval problem with minimal intervention to the models' architecture. 
Previous work has explored ways to partition the search space into hierarchical structures and retrieve documents by autoregressively generating their unique identifier.
In this work we propose an alternative that doesn't force any structure in the search space: using all ngrams in a passage as its possible identifiers. This setup allows us to use an autoregressive model to generate and score distinctive ngrams, that are then mapped to full passages through an efficient data structure.
Empirically, we show this not only outperforms prior autoregressive approaches but also leads to an average improvement of at least 10 points over more established retrieval solutions for passage-level retrieval on the KILT benchmark, establishing new state-of-the-art downstream performance on some datasets, while using a considerably lighter memory footprint than competing systems. Code and pre-trained models at \url{https://github.com/facebookresearch/SEAL}.
